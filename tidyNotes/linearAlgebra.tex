%You should title the file with a .tex extension (hw1.tex, for example)
\documentclass[12pt]{article}

\usepackage{fancyhdr, amsmath, enumerate, url, ulem, polynom, subfig,
  amssymb, amsthm, verbatim, graphicx,hyperref, booktabs, comment}
\usepackage{proof}

\oddsidemargin0cm
\topmargin-2cm     %I recommend adding these three lines to increase the 
\textwidth16.5cm   %amount of usable space on the page (and save trees)
\textheight23.5cm  

\newcommand{\question}[2] {\vspace{.25in} \hrule\vspace{0.5em}
\noindent{\bf #1: #2} \vspace{0.5em}
\hrule \vspace{.10in}}
\renewcommand{\part}[1] {\vspace{.10in} {\bf (#1)}}

\newcommand{\myname}{Yujie Xu}
\newcommand{\myandrew}{yujiex@andrew.cmu.edu}
\newcommand{\myhwnum}{00}

\newcommand{\fref}[1]{Figure \ref{#1}}
\newcommand{\tref}[1]{Table \ref{#1}}
\newcommand{\eref}[1]{Equation \ref{#1}}

\setlength{\parindent}{0pt}
\setlength{\parskip}{5pt plus 1pt}

 
\pagestyle{fancyplain}
\lhead{\fancyplain{}{\textbf{Lecture Note}}}      % Note the different brackets!
\rhead{\fancyplain{}{\myname\\ \myandrew}}
\chead{\fancyplain{}{15-456}}

%
% The following macro is used to generate the header.
%
\newcommand{\lecture}[4]{
   \pagestyle{myheadings}
   \thispagestyle{plain}
   \newpage
   \stepcounter{#1}
   \setcounter{page}{1}
   \noindent
   \begin{center}
      \vbox{\vspace{2mm}
       \vspace{4mm}
       \hbox to 6.28in { {\Large \hfill Lecture \arabic{#1}: #2  \hfill} }
       \vspace{2mm}
       \hbox to 6.28in { {\it \hfill Instructor: #3 \quad Notes taken: Yujie Xu\hfill}}      
       \hbox to 6.28in { {\it  \hfill Date: #4 \hfill} }
      \vspace{2mm}}
   \end{center}
   \markboth{Lecture #1: #2}{Lecture \arabic{#1}: #2}
}

\newcommand{\hw}[4]{
   \pagestyle{myheadings}
   \thispagestyle{plain}
   \newpage
   \stepcounter{#1}
   \setcounter{page}{1}
   \noindent
   \begin{center}
      \vbox{\vspace{2mm}
       \vspace{4mm}
       \hbox to 6.28in { {\Large \hfill HW \arabic{#1}: #2  \hfill} }
       \vspace{2mm}
       \hbox to 6.28in { {\it \hfill Instructor: #3 \quad Notes taken: Yujie Xu\hfill}}      
       \hbox to 6.28in { {\it \hfill Date: #4} }
      \vspace{2mm}}
   \end{center}
   \markboth{Lecture #1: #2}{Lecture \arabic{#1}: #2}
}
%
% The following is my command for comments
%
\newcommand{\claim}[1]{\par {\bf Claim }{#1}}
\newcommand{\subclaim}[1]{\par {\bf Subclaim }{#1}}
\newcommand{\hp}[1]{\par {\bf HP: }{#1}}
\newcommand{\note}[1]{\par {\bf Note: }{#1}}
\newcommand{\rmk}[1]{\par {\bf Remark: }{#1}}
\newcommand{\que}[1]{\par {\bf Question: }{\it #1}}
\newcommand{\ans}[1]{\par {\bf Answer: }{\it #1}}
\newcommand{\eg}[1]{\par {\bf Examples: }{#1}}
\newcommand{\wn}[1]{\par {\bf Warning: }{#1}}

\begin{document}

\medskip                        % Skip a "medium" amount of space
                                % (latex determines what medium is)
                                % Also try: \bigskip, \littleskip

%
% The following commands set up the lecnum (lecture number)
% counter and make various numbering schemes work relative
% to the lecture number.
%
\newcounter{lec}
\renewcommand{\thepage}{\thelec-\arabic{page}}
\renewcommand{\thesection}{\thelec.\arabic{section}}
\renewcommand{\theequation}{\thelec.\arabic{equation}}
\renewcommand{\thefigure}{\thelec.\arabic{figure}}
\renewcommand{\thetable}{\thelec.\arabic{table}}

% Use these for theorems, lemmas, proofs, etc.
\newtheorem{theorem}{Theorem}[lec]
\newtheorem{lemma}[theorem]{Lemma}
\newtheorem{uf}[theorem]{Useful Fact}
\newtheorem{proposition}[theorem]{Proposition}
\newtheorem{corollary}[theorem]{Corollary}
\newtheorem{definition}[theorem]{Definition}
\newtheorem{remark}[theorem]{Remark}
\newtheorem{recall}[theorem]{Recall}
\newtheorem{fact}[theorem]{Fact}
%\newenvironment{proof}{{\bf Proof:}}{\hfill\rule{2mm}{2mm}}
%
% Convention for citations is authors' initials followed by the year.
% For example, to cite a paper by Leighton and Maggs you would type
% \cite{LM89}, and to cite a paper by Strassen you would type \cite{S69}.
% (To avoid bibliography problems, for now we redefine the \cite command.)
% Also commands that create a suitable format for the reference list.
\begin{comment}
\renewcommand{\cite}[1]{[#1]}
\def\beginrefs{\begin{list}%
        {[\arabic{equation}]}{\usecounter{equation}
         \setlength{\leftmargin}{2.0truecm}\setlength{\labelsep}{0.4truecm}%
         \setlength{\labelwidth}{1.6truecm}}}
\def\endrefs{\end{list}}
\def\bibentry#1{\item[\hbox{[#1]}]}
\end{comment}

%Use this command for a figure; it puts a figure in wherever you want it.
%usage: \fig{NUMBER}{SPACE-IN-INCHES}{CAPTION}
\newcommand{\fig}[3]{
			\vspace{#2}
			\begin{center}
			Figure \thelecnum.#1:~#3
			\end{center}
}

\par Template adapted from CMU's 10725 Fall 2012 Optimization course
taught by Geoff Gordon and Ryan Tibshirani.
%%

\pagebreak
\title{Linear Algebra Review}
\maketitle
\lecture{lec}{Intro}{}{Monday, Aug 31, 2015}
\setcounter{section}{0}
\section{Concepts}
\begin{definition}
  Column Space: In linear algebra, the column space C(A) of a matrix A
  (sometimes called the range of a matrix) is the set of all possible
  linear combinations of its column vectors.
\end{definition}
A matrix with n columns and m rows corresponds to n vectors in
m-dimensional space.

Linear equations lead to geometry of planes.  Intersection of planes
leads to lines: in three dimensions a line requires two equations; in
a n dimensional space it will require $n - 1$.

There are two ways of looking at a system of linear equation: from
rows or from columns.
\begin{itemize}
\item Row picture: Intersection of planes

  If looking at rows, each row is a plain and their intersection is
  the solution. For an n-equation-n-variable system of equations, each
  equation is a $n - 1$ dimensional plain in a n dimensional
  space. For example, $2u + v + w = 10$ is a 2D plain in a 3D space.

  The first equation produces an $(n - 1)$-dimensional plane in n
  dimensions, The second plane intersects it (we hope) in a smaller
  set of ``dimension $n - 2$.'' Assuming all goes well, every new
  plane (every new equation) reduces the dimension by one. At the end,
  when all n planes are accounted for, the intersection has dimension
  zero. It is a point, it lies on all the planes, and its coordinates
  satisfy all n equations. It is the solution!

\item Col picture: Combination of columns

  If looking at columns, the problem of solving the system of linear
  equations becomes to find the combination of the column vectors on
  the left side that produces the vector on the right side. The
  co-efficient of each variable is considered as a point in
  n-dimensional space, which is also a vector. The multipliers of
  vectors that does the job is also the point where in the row picture
  all plane intersects.
    
  \begin{definition}
    Linear combination of vectors: \\vectors are multiplied by numbers
    and then added. The result is called a linear combination.
  \end{definition}
    
\end{itemize}
\begin{definition}
  Singular cases: no solution or infinite solution 
\\Non-singular case:
  there is one solution to the system of equations.
\end{definition}
\begin{itemize}
\item No solution: Inconsistent equations results in no solution.

  For row picture, this happens when two or planes are parallel, or
  the intersection line of two planes is parallel to other planes.
    
  For column picture, this happens when all vectors are in the same
  plane but the target vector is not in the plane.
\item Infinite many solution

  For row picture, when all planes intersects in one line, or even,
  when all planes are the same, there are infinitely many solutions.
    
  For column picture, this happens when all vectors are in the same
  plane and the target vector is also in the same plane.
\end{itemize}
If the n planes have no point in common, or infinitely many points,
then the n columns lie in the same plane

\section{Gaussian Elimination}
The algorithm:
\begin{verbatim}
Subtracting multiples of equation 1 from equation 2 to n to eliminate
the first variable

Recurse on step 1 on the remaining
\end{verbatim}
The (forward) elimination produces a triangular system, the values on
the hypotenuse is called \emph{pivots}. By definition, pivots cannot
be zero.
\begin{definition}
  In the following example, there are 9 \emph{coefficients}, 3
  \emph{unknowns}, 3 \emph{right-hand sides}
  \begin{align*}
    2u + v + w &= 5\\
    4u - 6v &= -2\\
    -2u + 7v + 2w &= 9
  \end{align*}
\end{definition}
\begin{definition}
  Square Matrix: \\For a coefficient matrix, when the number of
  unknowns equals the number of equations.
\end{definition}
\begin{definition}inner product:\\
  The number produced from the multiplication of a row vector and a
  column vector.
\end{definition}
There are two ways of thinking about the multiplication of $Ax$: 
\begin{itemize}
\item n inner product, one for each row.
\item a combination of columns in A: Ax is a combination of the
  columns of A. The coefficients are the components of x
\end{itemize}
\rmk The notation $x = (2, 5, 0)$ is equivalent to $x = [2 \;5 \;0]^T$
\begin{definition}
  The identity matrix I has 1s on the diagonal and 0s everywhere else
\end{definition}
\begin{definition}
  The elementary matrix $E_{ij}$ subtracts $l$ times row j from row i.
\end{definition}
\begin{theorem}
  Matrix multiplication is associative: $(AB)C = A(BC)$\\
  Matrix multiplication is distributive: $A(B + C) = AB + AC$.
  Matrix multiplication is NOT commutative
\end{theorem}
\begin{theorem}
  The i, j entry of AB is the inner product of the ith row of A and
  the jth column of B.
  
  Every column of AB is a combination of the columns of A. Column i is
  the combination columns in A with the ith column in B as
  multipliers: column j of AB $=$ A times (column j of B)
  
  Each row of AB is a combination of the rows of B: 

  row i of AB $=$ (row i of A) times B.
\end{theorem}
\begin{definition}Row exchange matrix:\\
  $$\begin{bmatrix}
  0 & 1\\
  1 & 0\\
  \end{bmatrix}
  \begin{bmatrix}
  2 & 3\\
  7 & 8\\
  \end{bmatrix}
  = 
  \begin{bmatrix}
  7 & 8\\
  2 & 3\\
  \end{bmatrix}
  $$

\end{definition}
\rmk{the product of lower triangular matrices is again lower
  triangular}
\begin{definition}upper triangle: \\

  All entries below the diagonal are zero.
\end{definition}
\section{Summations}
\begin{itemize}
\item $1^2 + 2^2 + \dots + n^2 = {n(n + 1)(2n + 1) \over 6}$
\item 
\item 
\item 
\item 
\end{itemize}
\bibliographystyle{plain}
\pagebreak
\bibliography{Bibliography}
\end{document}