%You should title the file with a .tex extension (hw1.tex, for example)
\documentclass[12pt]{article}

\usepackage{fancyhdr, amsmath, enumerate, url, ulem, polynom, subfig, amssymb, amsthm, verbatim, graphicx,hyperref, booktabs, comment}
\usepackage{proof}

\oddsidemargin0cm
\topmargin-2cm     %I recommend adding these three lines to increase the 
\textwidth16.5cm   %amount of usable space on the page (and save trees)
\textheight23.5cm  

\newcommand{\question}[2] {\vspace{.25in} \hrule\vspace{0.5em}
\noindent{\bf #1: #2} \vspace{0.5em}
\hrule \vspace{.10in}}
\renewcommand{\part}[1] {\vspace{.10in} {\bf (#1)}}

\newcommand{\myname}{Yujie Xu}
\newcommand{\myandrew}{yujiex@andrew.cmu.edu}
\newcommand{\myhwnum}{00}

\newcommand{\fref}[1]{Figure \ref{#1}}
\newcommand{\tref}[1]{Table \ref{#1}}
\newcommand{\eref}[1]{Equation \ref{#1}}

\setlength{\parindent}{0pt}
\setlength{\parskip}{5pt plus 1pt}

 
\pagestyle{fancyplain}
\lhead{\fancyplain{}{\textbf{Lecture Note}}}      % Note the different brackets!
\rhead{\fancyplain{}{\myname\\ \myandrew}}
\chead{\fancyplain{}{15-456}}

%
% The following macro is used to generate the header.
%
\newcommand{\lecture}[4]{
   \pagestyle{myheadings}
   \thispagestyle{plain}
   \newpage
   \stepcounter{#1}
   \setcounter{page}{1}
   \noindent
   \begin{center}
      \vbox{\vspace{2mm}
       \vspace{4mm}
       \hbox to 6.28in { {\Large \hfill Lecture \arabic{#1}: #2  \hfill} }
       \vspace{2mm}
       \hbox to 6.28in { {\it \hfill Instructor: #3 \quad Notes taken: Yujie Xu\hfill}}      
       \hbox to 6.28in { {\it  \hfill Date: #4 \hfill} }
      \vspace{2mm}}
   \end{center}
   \markboth{Lecture #1: #2}{Lecture \arabic{#1}: #2}
}

\newcommand{\hw}[4]{
   \pagestyle{myheadings}
   \thispagestyle{plain}
   \newpage
   \stepcounter{#1}
   \setcounter{page}{1}
   \noindent
   \begin{center}
      \vbox{\vspace{2mm}
       \vspace{4mm}
       \hbox to 6.28in { {\Large \hfill HW \arabic{#1}: #2  \hfill} }
       \vspace{2mm}
       \hbox to 6.28in { {\it \hfill Instructor: #3 \quad Notes taken: Yujie Xu\hfill}}      
       \hbox to 6.28in { {\it \hfill Date: #4} }
      \vspace{2mm}}
   \end{center}
   \markboth{Lecture #1: #2}{Lecture \arabic{#1}: #2}
}
%
% The following is my command for comments
%
\newcommand{\claim}[1]{\par {\bf Claim }{#1}}
\newcommand{\subclaim}[1]{\par {\bf Subclaim }{#1}}
\newcommand{\hp}[1]{\par {\bf HP: }{#1}}
\newcommand{\note}[1]{\par {\bf Note: }{#1}}
\newcommand{\rmk}[1]{\par {\bf Remark: }{#1}}
\newcommand{\que}[1]{\par {\bf Question: }{\it #1}}
\newcommand{\ans}[1]{\par {\bf Answer: }{\it #1}}
\newcommand{\eg}[1]{\par {\bf Examples: }{#1}}
\newcommand{\wn}[1]{\par {\bf Warning: }{#1}}

\begin{document}

\medskip                        % Skip a "medium" amount of space
                                % (latex determines what medium is)
                                % Also try: \bigskip, \littleskip

%
% The following commands set up the lecnum (lecture number)
% counter and make various numbering schemes work relative
% to the lecture number.
%
\newcounter{lec}
\renewcommand{\thepage}{\thelec-\arabic{page}}
\renewcommand{\thesection}{\thelec.\arabic{section}}
\renewcommand{\theequation}{\thelec.\arabic{equation}}
\renewcommand{\thefigure}{\thelec.\arabic{figure}}
\renewcommand{\thetable}{\thelec.\arabic{table}}

% Use these for theorems, lemmas, proofs, etc.
\newtheorem{theorem}{Theorem}[lec]
\newtheorem{lemma}[theorem]{Lemma}
\newtheorem{uf}[theorem]{Useful Fact}
\newtheorem{proposition}[theorem]{Proposition}
\newtheorem{corollary}[theorem]{Corollary}
\newtheorem{definition}[theorem]{Definition}
\newtheorem{remark}[theorem]{Remark}
\newtheorem{recall}[theorem]{Recall}
\newtheorem{fact}[theorem]{Fact}
%\newenvironment{proof}{{\bf Proof:}}{\hfill\rule{2mm}{2mm}}
%
% Convention for citations is authors' initials followed by the year.
% For example, to cite a paper by Leighton and Maggs you would type
% \cite{LM89}, and to cite a paper by Strassen you would type \cite{S69}.
% (To avoid bibliography problems, for now we redefine the \cite command.)
% Also commands that create a suitable format for the reference list.
\begin{comment}
\renewcommand{\cite}[1]{[#1]}
\def\beginrefs{\begin{list}%
        {[\arabic{equation}]}{\usecounter{equation}
         \setlength{\leftmargin}{2.0truecm}\setlength{\labelsep}{0.4truecm}%
         \setlength{\labelwidth}{1.6truecm}}}
\def\endrefs{\end{list}}
\def\bibentry#1{\item[\hbox{[#1]}]}
\end{comment}

%Use this command for a figure; it puts a figure in wherever you want it.
%usage: \fig{NUMBER}{SPACE-IN-INCHES}{CAPTION}
\newcommand{\fig}[3]{
			\vspace{#2}
			\begin{center}
			Figure \thelecnum.#1:~#3
			\end{center}
}

\par Template adapted from CMU's 10725 Fall 2012 Optimization course
taught by Geoff Gordon and Ryan Tibshirani.
%%

\pagebreak
\title{Note of Computational Geometry}
\maketitle
\lecture{lec}{Introduction, The Line Intersection Problem using
  Sweepline}{Gary Miller}{Monday, Aug 31, 2015}
\setcounter{section}{0}
\section{Intro}
\subsection{Course description from the syllabus}

``How do you sort points in space? What does it even mean? This course
takes the ideas of a traditional algorithms course, sorting,
searching, selecting, graphs, and optimization, and extends them to
problems on geometric inputs. We will cover many classical geometric
construc- tions and novel algorithmic methods. Some of the topics to
be covered are convex hulls, Delaunay triangulations, graph drawing,
point location, geometric medians, polytopes, configuration spaces,
computational topology, approximation algorithms, and others. This
course is a natural extension to 15-451, for those who want to learn
about algorithmic problems in higher dimensions.''

Textbook: Computational Geometry: Algorithms and
Applications~\cite{CG2008}.

Traditional algorithm courses mainly discuss 1D problems, such as
BST. The topics of this course contains the following main topics:
\begin{itemize}
\item Large dimensional problems
\item The change of the nature of simple geometry problems when
  dimension increases
\end{itemize}
The applications of computational geometry include:
\begin{itemize}
\item 2D: graphics
\item high dimension: machine learning
\end{itemize}
The basic issues or standard computational problems that will be
discussed in recent lectures include:
\begin{itemize}
\item Line segment intersection (sweepline algorithm, random
  incremental algorithm)
\item Convex Hull: given a set of points, compute the convex hull of
  these points.
\item 2D-LP
\end{itemize}
The standard geometry problems that will be discussed recently
include:
\begin{itemize}
\item Line side test
\item In circle test
\end{itemize}
First the abstract objects and their representation that will be used
in this course are discussed and is listed in the \tref{tab:AO}
\begin{table}[h!]
\centering
\caption{Abstract Object and Their Representation}
\label{tab:AO}
\begin{tabular}{@{}l p{5cm}l@{}}
\toprule
Absract Object & Representation                                                                        & Issue          \\ \midrule
Real Number    & Float                                                                                 & Rounding Err   \\
               & Bignum (with arbitrary precision, normally they use arbitrary length array of digits) & Memory Intense \\
               & Computer Algebra (Symbolic Computation)                                               &                \\
Point          & Pair of Real                                                                          &                \\
Line           & Pair of Points                                                                        &                \\
Line Segment   & Pair of Points                                                                        &                \\
Triangle       & Tripple of Points                                                                     &                \\ \bottomrule
\end{tabular}
\end{table}
\section{How to use points to generate object}
Suppose $P_1, P_2, \dots P_k \in \omega^d$ where $P_k \in M$ is a
d-dimensional vector space with each point being a $M$-dimensional
point, the following combinations of points creates the linear
subspace of the vector space and generates geometric objects:
\begin{itemize}
\item \emph{Linear Combination}: 
$$Subspace = \sum_{i} \alpha_i \cdot P_i, \alpha_i \in \mathbb{R}$$ 

For the $d = 2$ and $P_i \in \mathbb{R}^3$ case, the linear
combination of $P_1$ and $P_2$ forms a plane with
$\vec{OP_1}, \vec{OP_2}$ being the basis.

\item \emph{Affine Combination}: 
$$Plane = \sum_{i} \alpha_i \cdot P_i, s.t. \; \alpha_i \in \mathbb{R} \land
\sum_{i} \alpha_i = 1$$ 

For the $d = 2$ and $P_i \in \mathbb{R}^3$ case, the affine
combination of $P_1$ and $P_2$ forms a line that passes $P_1, P_2$.

\item \emph{Convex Combination}: 
$$Body = \sum_{i} \alpha_i \cdot P_i, s.t. \; \alpha_i \in \mathbb{R} \land
\sum_{i} \alpha_i = 1 \land \alpha_i \geq 0$$

$S \subseteq \mathbb{R}^d$ is a convex set iff
$\forall p, q \in S \;\ldotp, [p, q] \subseteq S$ (A set S in a vector
space over R is called a convex set if the line segment joining any
pair of points of S lies entirely in S~\cite{convexSetWA})

For the $d = 2$ and $P_i \in \mathbb{R}^3$ case, the convex
combination of $P_1$ and $P_2$ forms a line segment between
$P_1, P_2$.

\emph{Convex Closure/Hull: } The minimal convex set $S' \supseteq S$ 

A subset $S$ of the plane is called convex if and only if for any pair
of points $p, q \in S$ the line segment $\overline{pq}$ is completely
contained in $S$. The convex hull CH(S) of a set S is the smallest
convex set that contains S~\cite{CG2008}.

\begin{theorem}
  $CC(P_1 \dots P_k) = \{\alpha \in \mathbb{R} \mid \sum_{i} \alpha_i
  = 1 \land \alpha_i \geq 0\}$

  In convex geometry Carathéodory's theorem states that if a point x
  of $\mathbb{R}^d$ lies in the convex hull of a set $P$, there is a
  subset $P'$ of $P$ consisting of $d + 1$ or fewer points such that x
  lies in the convex hull of $P'$.

\end{theorem}
\end{itemize}
\section{Primitives}
Problems related to geometry primitives
\begin{enumerate}[1)]
\item Test equality $p = q?$
\item Line segment intersection in 2D

  Let $L_1 = [P_1, P_2]$, $L_2 = [P_3, P_4]$, let
  $P_i = \begin{pmatrix}x_i \\ y_i\end{pmatrix}$.

  \begin{equation}\label{eq:lineSeg1}
    L_1 \cap L_2 \neq 0 \iff (P_1 P_2)\begin{pmatrix}\alpha_1 \\
      \alpha_2\end{pmatrix} = (P_3 P_4)\begin{pmatrix}\alpha_3 \\
      \alpha_4\end{pmatrix}
    \\\text{ and } \alpha_1 + \alpha_2 = \alpha_3 + \alpha_4 = 1 \text{ and }
    \alpha_i \geq 0.
  \end{equation}
  If written in matrix form, \eref{eq:lineSeg1} becomes:
  \begin{equation}\label{eq:lineSeg2}
    \begin{pmatrix}
      x_1 & x_2 & -x_3 & -x_4 \\
      y_1 & y_2 & -y_3 & -y_4 \\
      1 & 1 & 0 & 0 \\
      0 & 0 & 1 & 1 
    \end{pmatrix}
    \cdot
    \begin{pmatrix}
      \alpha_1\\
      \alpha_2\\
      \alpha_3\\
      \alpha_4\\
    \end{pmatrix}
    = 
    \begin{pmatrix}
      0\\
      0\\
      1\\
      1\\
    \end{pmatrix}
    \end{equation}
    So the general process of solving the line segment intersection
    problem is
    \begin{enumerate}[{Step }I.]
    \item Solve{eq:lineSeg2} with some solver
    \item Check if $\forall i, \alpha_i \geq 0$
    \end{enumerate}
    There could be more than one solutions when $L_1$ and $L_2$
    intersect in more than one point: the four point are collinear and
    there is an overlay. But in terms of the problem, the solution is
    still unique, since the solution is either True of False.
    
    \rmk{The good practice is make this line segment intersection test
      be a sum-routine and call an existing solver to solve the
      matrix. Don't try to inline the code}
    
    Some cases when the algorithm output False:
    \begin{itemize}
    \item $L_1$ and $L_2$ parallel but not collinear
    \item the intersection is on the extension of the two line segment
    \item the intersection is on one line segment and on the extension
      of the other.
    \end{itemize}

\item Line side test
  \begin{itemize}
  \item input: three points in 2D: $P_1, P_2, P_3$
  \item output: if $P_3$ is to the left of ray ${P_1P_2}$ 

    One process of solving the problem: Subtract $P_1$ from both of the
    other vectors.  Let $V_2 = P_2-P_14$ and $V_3 = P_3-P_1$.  Now the
    cross product $V_2 \times V_3$ is the signed area of the
    parallogram formed by $V_2$ and $V_3$.  This area is $>$ 0 if and
    only if $P_3$ is to the left of ray $P_1 \mapsto P_2$.
    
    Alternatively, suppose $P_1 = O$ (the origin), then the signed
    area of the parallogram formed by $\vec{P_1 P_2}, \vec{P_1, P_3}$
    is 
    \begin{equation}\label{eq:signArea}
      \det\begin{pmatrix}
        x_2 & x_3 \\
        y_2 & y_3 \\
      \end{pmatrix}
    \end{equation}

    \begin{equation}\label{eq:lineSideDet}
      LHS = \det\begin{bmatrix}
        x_1 & x_2 & x_2\\
        y_1 & y_2 & y_2\\
        1 & 1 & 1\\
      \end{bmatrix}
      =
      det\begin{bmatrix}
        x_1 & x_2 - x_1 & x_3 - x_1\\
        y_1 & y_2 - y_1 & y_3 - y_1\\
        1 & 0 & 0\\
        \end{bmatrix}
    \end{equation}
    Just report the sign of determinant of LHS of
    \eref{eq:lineSideDet}
  \end{itemize}
  \rmk{The good property of method 2 is that it can be generalized to
    higher dimension easily: for a 4D space, the test checks the
    determinant of a 4 by 4 matrix.}

\item In circle test
  \begin{itemize}
  \item input: four points in 2D: $P_1, P_2, P_3, P_4$
  \item output: if $P_4$ is in the circle of $(P_1, P_2, P_3)$
  \end{itemize}
\end{enumerate}
\bibliographystyle{plain}
\pagebreak
\bibliography{Bibliography}
\end{document}